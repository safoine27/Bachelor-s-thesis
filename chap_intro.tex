\addcontentsline{toc}{chapter}{Background and motivation}
\chapter*{Background and motivation}
\label{Intro}
 
\lettrine[lines=3,loversize=0.5]{T}{}he Web is a major technology in 2\nth{1} century. Its birth came after Internet and hypertext combination. Over the past decades, the use of the Web have changed a lot - it has influenced our social and commercial activities. In the beginning, Web was static - it transmits information using static HTML (Hypertext Markup Language) documents. After few years, new Web languages have been created like CSS (Cascading Style Sheets), JavaScript and much more in order to improve the performance of the Web .
With the appearance of social media, e.g. Facebook, Twitter, Youtube, that have been invading the Web and have eased the set of personal relations. The use of internet has continued to rise and become a daily routine for most of people.
At the beginning of the 2000's, Web have ameliorate the work of search engines by organizing the big amount of data considering their semantic by annotating documents .\\
New other languages were created like RDF (Resource Description Framework), RDFS (Resource Description Framework Schema)
, SPARQL (SPARQL Protocol and RDF Query Language) that form the base and fundamental of semantic Web.\\



The semantic Web aims to develop a smart Web that holds resources which can be understood by both human and computers. It basically uses annotations and Ontologies. The word “Ontology” is used with different senses in different communities. The most radical difference is perhaps between the philosophical sense, which has of course a well-established tradition, and the computational sense, which emerged in the recent years in the knowledge engineering community, starting from an early informal definition of (computational) Ontologies as \textit{"explicit specifications of conceptualizations"}. Ontology definition in philosophy is \textit{"The study of being as a being"} but in computer science field is a model to represent the meaning of the knowledge of a domain. To represent an Ontology, W3C created a standard language OWL (Ontology Web Language), built on the RDF data model. It adds the ability to define classes in more complex connectors corresponding to the description logic. Ontology is therefore used to reason about the objects of a certain domain concerned.
The conception of an Ontology includes both building of a model of knowledge representation and the enrichment of that model (
attachment to the elements components the schema of instances coming from the resources of the cooperating systems). An ontology corresponds to a common controlled, organized, and shared vocabulary and to the explicit formalization of the relations created between the different terms of the vocabulary.\\

The works already carried out within the Semantic Web is limited to the processing of textual information. To date, there is no way to synthesize information in the form of dynamic graphics generated on demand and which conform to the data that accompanies them. Therefore, it exists the variable data that can change, so we need to generate a graphic when the user needs it. The semantic Web can not provide an semantic annotations for the on demand graphical representation of variable data in Web documents. \\ 

As an example, let us consider an election period, the French presidential election of 2017 for instance. People would like to have an idea about the trends and the scores that could be obtained by each candidate. A simple search on the Web may lead to a huge amount of textual data that could be impossible to manage as is. However, the best way to represent trends in general is a graphical representation.
At the present time, graphical representations of such trends are given by simple static images. However these images may be too old and do not represent the real trends. The graphic synthesis of the textual information is a way to facilitate the understanding and the manipulation of the data. Every dynamic or variable phenomenon may be represented graphically. The variable may be quantitative or qualitative. It may also be a stochastic variable. The graphical representation of a variable data must be consistent with the data source in order to be credible and effectual.

Our objective is to implement an approach that allows to automatically analyze documents in order to  extract context and relations and to provide a semantic annotation that means to enable the graphical representation on demand. \\


This report comports four chapters. In the first one, we present the evolution of Web and the technologies. In the second chapter, we discuss the birth of the semantic Web, its architecture and its basic bricks (technologies and languages). In the third chapter, we focus on the usefulness of graphical information synthesis and we illustrate theoretically our general process of graphical representation on demand. In the last chapter, we demonstrate the process's implementation. \\




