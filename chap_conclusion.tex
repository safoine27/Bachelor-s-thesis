\addcontentsline{toc}{chapter}{General conclusion}
\chapter*{General conclusion}
\addcontentsline{toc}{section}{Summary of the main contributions}
\section*{Summary of the main contributions}

The main function of W3C is to standardize the representation and exchange of information on the Web. This objective should help to make information comprehensible to automated processes and users. New standards have been developed to allow the semantic representation of information in the form of languages derived from XML. This evolution is called Semantic Web. \\

Many languages have been developed within the framework of the semantic Web and most of these languages are based on the XML language. The OWL language and the RDF language are very important languages of the semantic Web. The OWL language allows to represent the ontologies, and it proposes to the machines a great capacity of execution of the Web content. The RDF is the first W3C standard for Web resource enrichment with detailed descriptions. These languages are used to represent the semantics associated with information, whatever its form and structure in the form of graphs. \\

Unstructured data is data that does not follow a specified format for big data. If 20 percent of the data available to enterprises is structured data, the other 80 percent is unstructured. Unstructured data is really most of the data that you will encounter. Until recently, however, the technology didn’t really support doing much with it except storing it or analyzing it manually. \\


The main contribution of this work is to implement an "On demand graphical representation" of unstructured documents on the Web. Another contribution of this work is to develop a tool that analyze unstructured text documents on the Web and extract context and all its relations. The process of the most important phases of our work can be summarized in five steps: 
 \begin{itemize}
    \item \textbf{The analysis of document} :To determinate the context and relations.
    \item \textbf{The construction of an ontology}:The creation or not of an ontology depend of the context of the analyzed document.
    \item \textbf{The annotation of the document} :Generating an RDF annotation of the context ontology.
    \item \textbf{Constructing the graphical models} : Constructing an ontology that determines the possible representation models for each criteria of variability.
    \item \textbf{The data visualization} :Automatic on demand graphical representation of the variable data.
    
    \end{itemize}
\addcontentsline{toc}{section}{Perspectives}    
    \section*{Perspectives}
We have managed to implement our proposed approach, Although we had lots difficulties since there wasn't enough tools and technologies that we can use in order to ease up the implementation phase. Thus, we had to develop our own algorithms and frameworks which was a a bit hard process to begin with.
For the moment our work isn't perfect, since the text analysis isn't very effective which led us to remodel the ontology population into a semi-automatic procedure and indeed, the analysis process concerns only a single document for the moment. Our future research concerns the graphic representation of data collected from multiple documents, and improving the text analysis process.
